\documentclass[12pt]{article}
\usepackage{url}
\begin{document}
\author{Patryk Schneider}
\title{DA07 - Transhumanism}
\maketitle

\section{Abstract}
	In 1923, a British scientist John B.S. Haldane wrote an essay called "Daedalus\; or, Science and the Future" for a lecture read to the Heretics Society. In his paper he argued that science applied to human biology can greatly benefit it in various ways, even though such practices would be considered "indecent and unnatural" \cite{haldane} at first. This work is considered to be the foundation of transhumanism - a new philosophical idea, claiming that the technology will play pivotal role in human evolution in the future. Over the years, this concept has developed into intellectual and cultural movements. In this paper, I would like to give a brief overview about possible evolutionary paths considered currently by transhumanists.

\newpage
%%%%%%%%%%%%%%%%%%%% INTRODUCTION
\section{Introduction}
	\emph{Mankind}. \emph{Technology}. \emph{Progress}. Those three words are used to describe our own history. Humanity has always pursued technological advancements for various reasons - improving lives, shaping the world to our liking, just to name a few. So far, all the innovations haven't directly altered the biology of our species - the very fundamentals of what it means to be a human have remained the same for thousands of years. This, however, might change soon. 
	\\According to some studies, "a powerful acceleration of technological progress" is expected to happen somewhere "between the 2030s and the 2070s" \cite{progressRate}. This has left many people wondering - what will happen next? What is the next big step in the technological journey of our kind?
	\\Many futurists like Max More believe that it will be something that affects our very own bodies and minds. Something that aims to bypass limitations given to us by nature, fundamentally redefining what does it mean to be a human. That thing is technology.

\subsection{"Transhumanism" - history \& current definition}
	The idea of technology having an impact on human biology is relatively old - it has been almost over a century since it was first introduced by Haldane in his essay \cite{haldane}. During this time, term "transhumanism" was born and redefined multiple times. In my opinion, there are 3 important steps that led to its current state. To begin with, the very first definition of "transhumanism" comes from W. D. Lighthall - he considered it "a view of cosmic, biological, and cultural evolution"\cite{transhumanismHistoryLighthall}. Years later, Julian Huxley reformulated the term - according to his work, transhumanism is mankind's ability to "transcend itself - not just sporadically, an individual there in another way, but in its entirety, as humanity." \cite{transhumanismHistoryHuxley}. The last big change came with the new generation of transhumanists. An actual movement was born and started gaining momentum across the world. Various transhumanist associations were created and among them was a non-profit organization called \emph{Humanity+} (formally \emph{World Transhumanist Association}). Main goal of Humanity+ is spreading the idea and educating others on the subject. In order to do that, a modern formal definition of transhumanism was needed. Humanity+ revisited the term and redefined it to mean two things: 
	\\- "intellectual and cultural movement", which advocates for human enhancement through usage of technology;
	\\- study of possibilities, dangers and ethical implications created by development and usage of such technologies \cite{transhumanistFAQ:1}.
	\\The term itself is a combination of two expressions - \emph{trans} and \emph{humanism}. The former used as a prefix, coming from Latin, means simply \emph{beyond} \cite{transTermDictionary}, while the latter, refers to a philosophy which focuses on freedom, dignity and welfare of human beings \cite{humanism:1}. Those two words, combined, form (in literal meaning) \emph{beyond-human}. 

\subsection{The Posthuman - a human beyond}
	The \emph{Posthuman} is the answer for transhumanism's question "how will technology change the humans?". It is the next step in the evolution of the mankind (which has reached a standstill ever since \emph{Homo sapiens} have emerged around 315 thousands years ago \cite{homosapiens:1}). 
	\\How exactly do we define a \emph{Posthuman}? According to Nick Bostrom, a posthuman is "a being that has at least one posthuman capacity" \cite{posthuman:1}. Those "posthuman capacities" are quite literally regular human capacities that have been transcended beyond current (natural) limitations. Lifespan, health, cognitive abilities, intelligence - just to name a few. Those things have to be taken to a whole new level if humanity wants to evolve beyond what we currently are. 
	\\One important thing about the term \emph{posthuman} is the fact that it has no strict, detailed definition (yet). Posthumans could be synthetic organisms that have both their own "human" intelligence and artificial intelligence. They could be just like us but have their bodies redesigned with nanotechnology in order to prevent things like ageing. Humanity could become a big network of interconnected human minds (\emph{The Global Brain}), forming sort of global "hive mind" \cite{transhumanistFAQ:2}. The possibilities are quite literally infinite at this point in time. We do not know what technological advancements will be made in the next century - we can really only try to predict it based on current research that is being made towards the goal of ascending humanity into a new kind of beings.
%%%%%%%%%%%%%%%%%%%% MIND
\section{Mind, redefined}

\subsection{Motivation behind mind enhancements}
	Brain is the most important organ in the body of a human. It is the control centre of our bodies - this is where all the life processes are managed and where every decision is made \cite{brain:1}. It is a powerhouse. Unfortunately, a fragile one. While it is true that brain is protected by the skull, it is still vulnerable to various kinds of injury. For example, all it takes is four minutes without oxygen to cause permanent damage \cite{lackofoxygen}. All it takes is a slightly reduced blood supply to cause death of brain cells (commonly known as a stroke) \cite{stroke:1}. The worst thing about it, however, is the fact that human brain is slowly deteriorating over time (according to studies "it has been widely found that the volume of the brain and/or its weight declines with age at a rate of around 5\% per decade after age 40 with the actual rate of decline possibly increasing with age particularly over age 70" \cite{brainAging}), becoming more and more vulnerable to cognitive disorders such as dementia.
	\\Now, imagine if there was a possibility to repair or replace damaged parts of the brain (or in general, nervous system). Imagine if we went even further, and we were able to create synthetic brain that is able to mimic the functionality of a "biological" brain completely. Finally, imagine if we could move the "data" from biological brain to synthetic one with minimal or even without any loss. Many people consider such things a "science-fiction" or dreams. However, transhumanists and scientists all around the globe have been working on them for years.

\subsection{Importance of multiplatform}
	The first step in recreating the brain in a synthetic form is the creation of \emph{substrate-independent mind} (SIM). What is it? To put it quite simply, it is a special type of mind, where "its selfsame functions that represent thinking process can be implemented through the operations available in a number of different computational platforms"\cite{sim:1}. It means that a mind is a SIM if the same function of a mind can be performed by a biological one and a synthetic one (computer hardware/software). Why exactly do we need it? We already have a working mind that "runs" on our biological brains. Well, it works as long as our brain stays functional. However, should we want to migrate our minds to a synthetic brain one day to either preserve it or replace damaged part, we will have to find a way to "continue personality, individual characteristics, a manner of experiencing, and a personal way of processing those experiences"\cite{sim:2}\cite{sim:3}. This is precisely SIM's goal - finding a general way of simulating unique human mind that is "hardware-independent". It is also the reason why we haven't achieved it yet - the scope is huge. \emph{Mind simulation} presents a challenge for people in all kinds of fields: philosophy, neuroscience, computer science - just to name a few.
	\\One is certain - it is definitely possible. Individual problems are already being worked on, but it will require time and combined effort of many people to make it real.

\subsection{Whole Brain Emulation}
	SIM in general focuses on replicating the phenomenon of "being". This however, is only the beginning. The next milestone is \emph{emulation} of the entire brain one-to-one. According to "Whole Brain Emulation Roadmap" technical report, "the basic idea is to take a particular brain, scan its structure in detail, and construct a software model of it that is so faithful to the original that, when run on appropriate hardware, it will behave in essentially the same way as the original brain."\cite{wbe:2}. What is more, the authors believe that "in order to emulate the brain we do not need to understand the whole system, but rather we just need a database containing all necessary low‐level information about the brain and knowledge of the local update rules that change brain states from moment to moment".\cite{wbe:2}. This is a great news - we don't need all the details to make it work. Are we \emph{really} ready, though? We might have the knowledge required to achieve it, but is our technology ready for such a simulation? Do we have enough computing power to run WBE? Apparently, we do. Basic way to measure required performance is to find out how many synapses are in our brains and how many operations they perform. Those numbers are as following:
	\\- 10 impulses per second;
	\\- ~10\textsuperscript{15} synapses in brain \cite{uploading:3}.\\This gives an estimate of 10\textsuperscript{16} operations per second. Supercomputers are already performing better than that - world's fastest supercomputer, Fujitsu Fugaku, is reaching performance of 3.9 exaflops \cite{supercomputer:1} (which means it's capable of performing 3.9 * 10\textsuperscript{18} 8-bit operations per second). Tech already exists - all that is left is finding a proper way to use it.

\subsection{Memories transferred}
	SIM combined with WBE will make artificial brains a reality. Only one last thing remains - how do we transfer our minds from one "vessel" to another? Ralph C. Merkle points out that brain is "a material object". Therefore, it is susceptible to laws of physics - which can be "modeled on a computer" \cite{uploading:1} (that computer can be Whole Brain Emulator, for example). This allowed him to create a simple "recipe" for transferring the mind from one brain to another:
	\\- scan every atom in the original brain (its location in 3D space);
	\\- save the data about every single atom;
	\\- move the data onto the new brain;
	\\- launch a physics model simulation on the new brain (based on the position data of the atoms in the original brain) \cite{uploading:1}.
	\\While this doesn't sound complicated, there is one thing to consider: how much space would the data about \emph{every single atom} in human brain take? 
	\\According to some early calculations presented by Merkle, to describe 1 atom we need around 100 bits on average. Human brain is built from around 10\textsuperscript{26} atoms. This gives us a total of 10\textsuperscript{28} bits required to describe the brain \cite{uploading:1}. We are talking about around 10\textsuperscript{14} \emph{terabytes} of data. This sounds impossible as of right now, but it might not be in the future. According to statistics, data centre storage capacity in 2021 came to the level of around 2300 \emph{exabytes} \cite{statista:capacity}. This is roughly 2.3*10\textsuperscript{9} \emph{terabytes}. It is sufficient to say that storage technology is slowly reaching the capacity required for storage of a human brain in a form of data about atoms. When it does, we will be able to start working on arguably one of the most important journeys in human history - a mind's journey from one brain to another.
%%%%%%%%%%%%%%%%%%%% AI
\section{AI and the human evolution}
\subsection{The AGI}
	Alan Turing asked an important question in his paper in 1950 - \emph{"can machines think?"} \cite{turing}. After 70 years of research in the field of AI, we are slowly reaching the conclusion: \emph{yes}, they can \emph{think}. Today, in 2021, there are many ways to categorize AI, but one of the most general distinctions is \emph{weak AI} (\emph{narrow AI}) and \emph{strong AI} (often called \emph{Artificial General Intelligence}). The main difference is that the former is supposed to be capable of performing one task, while the latter is supposed to match human intelligence - including ability to "solve problems, learn and plan for the future" \cite{agi:1}. Obviously, the AGI is the thing that transhumanists focus on - machine creation that is capable of \emph{thinking}, \emph{learning} and \emph{understanding} on its own could have an impact on the direction of the humanity (and the world).

\subsection{Possible influence of the AGI on our lives}
	Creation of an intelligence "similar to human, but different" is important for mankind, because it will offer a new point of view. Throughout the history of the mankind, different cultures, religions and beliefs have interacted with each other and influenced mindsets of millions of people. There is one problem about it - all of them were created by humans. Creatures that shared one common trait - being mortal lifeforms built from flesh. This is where the AGI could change the world - artificial beings could offer us new ways of looking at the world. Another point worth making is that the idea of AGI is that it is \emph{at least} as smart as we are. Therefore, it should be able to learn and understand things that a regular human can. Therefore, it can assist us in performing science research in various areas - just imagine if the first "intelligence" to find a proof (or counter-proof) to Riemann hypothesis (one of the Millennium Prize Problems) does not belong to a \emph{human}.

\subsection{The Singularity}
	In previous paragraphs I mentioned that AGI should be \emph{at least smart as we are}. This implies that if we are able to create intelligence on our level or higher, the AGI should be able to do the same (and most certainly will). This is the concept of \emph{the Singularity} - "technological explosion" - when "machines could design even better machines" \cite{singularity:1} - it also includes AI creating new, more intelligent, version of itself. This will start a (possibly) never-ending cycle of self-improvement of the AGI. 
	\\There are three basic questions asked about the Singularity: "will it actually happen?", "when will it happen?" and "what will be the outcome?". Philosopher David Chalmers argues in his essay that the Singularity is very likely to happen as long as one condition is met: humanity is able to create an AI more intelligent than itself. This will prove that one intelligence can create a better one \cite{singularity:3}. In the same essay, he claimed that it has already happened - "evolution produced human-level intelligence". Because of that, Chalmers is confident that "human-level intelligence can produce AI" \cite{singularity:3} that matches our own intelligence and then one that is more intelligent than us - which will lead to the Singularity. 
	\\There have been many speculations about \emph{when} and \emph{how fast} it will happen - is it coming in the near or distant future? Will it be one big "boom" or will it happen gradually over the years? We do not know - we can only speculate. Chalmers believes that the Singularity will happen in the next few hundred years \cite{singularity:2}\cite{singularity:3}, while Ray Kurzweil gives a more precise date - somewhere around 2045 \cite{singularity:2}\cite{singularity:4}.
	\\We already know that the Singularity will most likely happen. We also have some early predictions on the date. One question still remains: how will the "intelligence explosion" affect the humanity? One possibility is that the superintelligence created by the Singularity with provide solutions to modern-day problems - "a cure for all known diseases, an end to poverty, extraordinary scientific advances" \cite{singularity:3}. The other possibilities are rather grim: what if it decides that humanity is dangerous? What if someone uses it for bad reasons? 

\subsection{The Superbeing and The Global Brain}
	Concept of many things, interconnected into one big thing, is nothing new. It already exists in the nature - single-cell organisms combining into multi-cell organisms to evolve into something new or the "hive minds" ("swarm intelligence" -  "systems composed of many individuals that coordinate using decentralized control and self-organization" \cite{swarm}) are just a few examples of that. Some people believe that the next step for humanity is the so-called \emph{metasystem transition} \cite{turchin:1}, which means "evolution of a higher level of control and cognition". \cite{heylighen:1}. The goal is simple: \emph{integration} into one \emph{superbeing}, which will communicate through technology connected directly to nervous system of unique organisms. Does it mean that some scientists want to merge our minds into one? No, not exactly. There is a reason why I mentioned swam intelligence a few sentences ago. Honeybees might be interconnected into one big "hive mind", but each of them is its own being. The same idea would be applied here - each human would be his own \emph{unique} being, but possibly all humans would be "cloud-connected" into one big network with the rest. Another way to look at it would be Ben Goertzel's idea where some kind of advanced AGI will function as the main component of The Global Brain and human minds (alongside weaker AIs) would be able to connect to that network \cite{globalbrain:1}, essentially merging our human intelligence with (various kinds of) the AI.
%%%%%%%%%%%%%%%%%%%% SUMMARY	
\section{Summary and discussion} 
	In this paper I have shown some of the ideas that have been advocated for by (modern) transhumanists. Those proposals sound as if they were taken from "sci-fi" novels, but in reality there are many scientists who are already researching them. For the purpose of this paper, I have decided to focus on possible enhancements of human mind through technology and AI. In my opinion, the first step of evolving into posthumans will be bypassing the limitations given to us by nature - especially things like brain's vulnerability to ageing, permanent damage or its natural intelligence restrictions. This is why I have decided to showcase ideas of Substrate-Independent Mind, Whole Brain Emulation, Uploading, Artificial General Intelligence, The Singularity and the Global Brain ideas. All of them, combined, are possible alternatives and upgrades for a fragile powerhouse that is our brain. All of them could fundamentally redefine what does it mean to be a human. There are still aspects that I haven't covered - such as enhancements to the rest of human body, threats presented by those ideas or moral implications. 
	\\I believe that we live in interesting times - and that even better are ahead of us. Quite literally, history is being made right in front of our eyes - technological progress has been rapidly changing the world around us over the last 100 years and rate of progress is expected to accelerate even further in the near future. This is both exciting and dangerous - we won't know the outcome of those changes until they have actually happend (and by then it will be too late to stop them). Right now, we can only try to predict the directions that human evolution will follow. This is why I find \emph{Transhumanism} fascinating - it is not only "daydreaming" about the future, but also actual research on possibilities offered by the technology.

\raggedright
\bibliography{transhumanism-references}
\bibliographystyle{unsrt}
\end{document}