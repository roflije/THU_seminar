\documentclass[12pt]{article}
\usepackage{url}
\begin{document}
\author{Patryk Schneider}
\title{DA07 - Transhumanism}
\maketitle

\section{Abstract}
	In 1923, a British scientist John B.S. Haldane wrote an essay called " Daedalus\; or, Science and the Future " for a lecture read to the Heretics Society. In his paper he argued that science applied to human biology can greatly benefit it in various ways, even though such practices would be considered "indecent and unnatural" \cite{haldane} at first. This work is considered to be the foundation of transhumanism - a new philosophical idea, claiming that the technology will play pivotal role in human evolution in the future. Over the years, this concept has developed into intellectual and cultural movements. In this paper, I would like to give a brief overview about possible evolutionary paths considered currently by transhumanists.

\newpage

\section{Introduction}
	\emph{Mankind}. \emph{Technology}. \emph{Progress}. Those three words are used to describe our own history. Humanity has always pursued technologial advancements for various reasons - improving lives, shaping the world to our liking, just to name a few. So far, all the innovations haven't directly altered the biology of our species - the very fundamentals of what it means to be a human have remained the same for thousands of years. 
	\\This, however, might change soon. According to some studies, "a powerful acceleration of technological progress" is expected to happen somewhere "between the 2030s and the 2070s"  \cite{progressRate}. This has left many people wondering - what will happen next? What is the next big step in the technological journey of our kind?
	\\Many futurists like Max More believe that it will be something that affects our very own bodies and minds. Something that aims to bypass limitations given to us by nature, fundamentally redefining what does it mean to be a human. That thing is technology.

\subsection{"Transhumanism" - history \& current definition}
	The idea of technology having an impact on human biology is relatively old - it has been almost over a century since it was first introduced by Haldane in his essay \cite{haldane}. During this time, term "transhumanism" was born and redefined multiple times\cite{transhumanismEarlyThinking}. In my opinion, there are 3 important steps that led to its current state. To begin with, the very first definition of "transhumanism" comes from W. D. Lighthall - he considered it "a view of cosmic, biological, and cultural evolution"\cite{transhumanismHistoryLighthall}. Years later, Julian Huxley reformulated the term - according to his work, transhumanism is mankind's ability to "transcend itself - not just sporadically, an individual there in another way, but in its entirety, as humanity." \cite{transhumanismHistoryHuxley}. The last big change came with the new generation of transhumanists. An actual movement was born and started gaining momentum across the world. Various transhumanist associations were created and among them was a non-profit organization called \emph{Humanity+} (formally \emph{World Transhumanist Association}). Main goal of Humanity+ is spreading the idea and educating others on the subject. In order to do that, a modern formal definition of transhumanism was required. Humanity+ revisited the term and redefined it to mean two things: 
	\\- "intellectual and cultural movement", which advocates for human enhancement through usage of technology;
	\\- study of possibilities, dangers and ehtical implications created by development and usage of such technologies \cite{transhumanistFAQ:1}.
	\\The term itself is a combination of two expressions - \emph{trans} and \emph{humanism}. The former used as a prefix, coming from Latin, means simply \emph{beyond} \cite{transTermDictionary}, while the latter, according to Wikipedia, "refers to a focus on human well-being and advocates for human freedom, autonomy, and progress" \cite{humanismTermWiki}. Those two words, combined, form (in literal meaning) \emph{beyond-human}. 

\subsection{The Posthuman - a human beyond}
	Transhumanism, quite literally, means \emph{beyond human}. 
	todo(finish)

\section{Mind, redefined}
\subsection{Motivation behind mind enhancements}
	Brain is the most important organ in the body of a human. It is a center of the nervous system, responsible for most body activities, processing "data" from sense organs and making decisions for the rest of the body \cite{humanBrainWiki}. It is a powerhouse. Unfortunately, a fragile one. While it is true that brain is protected by the skull, it is still vulnerable to various kinds of injury. For example, all it takes is four minutes without oxygen to cause permanent damage \cite{lackofoxygen}. All it takes is a poor blood flow to cause death of cells (commonly known as a stroke)\cite{wikipediaStroke}. The worst thing about it, however, is a fact that human brain is slowly deteriorating over time (according to studies "it has been widely found that the volume of the brain and/or its weight declines with age at a rate of around 5\% per decade after age 40 with the actual rate of decline possibly increasing with age particularly over age 70" \cite{brainAging}), becoming more and more vulnerable to cognitive disorders such as dementia.
	\\Now, imagine if there was a possibility to repair or replace damaged parts of the brain (or in general, nervous system). Imagine if we went even further, and we were able to create synthetic brain that is able to mimic the functionality of a "biological" brain completely. Finally, imagine if we could move the "data" from biological brain to synthetic one with minimal or even without any loss. Many people consider such things a "science-fiction" or dreams. However, transhumanists and scientists all around the globe have been working on them for years.
\subsection{Multiplatform first}
	The first step in recreating the brain in a synthetic form is the creation of \emph{substrate-independent mind} (SIM). What is SIM? To put it quite simply, it is a special type of mind, where "its selfsame functions that represent thinking process can be implemented through the operations available in a number of different computational platforms"\cite{sim:1}. It means that a mind is a SIM if the same function of a mind can be performed by a biological one and a synthetic one (computer hardware/software). Why exactly do we need it? We already have a working mind that "runs" on our biological brains. Well, it works as long as our brain stays natural. However, should we want to migrate our minds to a synthetic brain one day (doesn't even have to be a full one, even just a part of it!), we will have to find a way to "continue personality, individual characteristics, a manner of experiencing, and a personal way of processing those experiences"\cite{sim:2}\cite{sim:3}. 
	\\This is precisely SIM's goal - finding a general way of simulating unique human mind. It is also the reason why we haven't achieved it yet - the scope is huge. \emph{Mind simulation} presents a challenge for people in all kinds of fields: philosophy, neuroscience, computer science - just to name a few.
	\\One is certain - it is definitely possible. Individual problems are already being worked on, but it will require time and combined effort of many people to make it real.
\subsection{Whole Brain Emulation}
	SIM in general focuses on replicating the phenomenon of "being". This however, is only the beginning. The next milestone is \emph{emulation} of the entire brain one-to-one. According to "Whole Brain Emulation Roadmap" technical report, "the basic idea is to take a particular brain, scan its structure in detail, and construct a software model of it that is so faithful to the original that, when run on appropriate hardware, it will behave in essentially the same way as the original brain."\cite{wbe:2}. What is more, the authors believe that "in order to emulate the brain we do not need to understand the whole system, but rather we just need a database containing all necessary low‐level information about the brain and knowledge of the local update rules that change brain states from moment to moment".\cite{wbe:2}. This is a great news - we don't need all the details to make it work. Are we \emph{really} ready, though? We might have the knowledge required to achieve it, but is our technology ready for such a simulation? Do we have enough computing power to run WBE? Apparently, we do. Basic way to measure required performance is to find out how many synapses are in our brains and how many operations they perform. Those numbers are as following:
	\\- 10 impulses per second;
	\\- ~10\textsuperscript{15} synapses in brain \cite{uploading:3}.\\This gives an estimate of 10\textsuperscript{16} operations per second. Supercomputers are already performing better than that - world's fastest supercomputer, Fujitsu Fugaku, is reaching performance of 442,010 teraflops per second \cite{wikipediaSupercomputer} (which means it's capable of performing 4.42010 * 10\textsuperscript{17} floating-point operations per second).
\subsubsection{Memories transferred}
	SIM combined with WBE will make artifical brains a reality. Only one last thing remains - how do we transfer our minds from one "vessel" to another? Ralph C. Merkle points out that brain is "a material object". Therefore, it is susceptible to laws of physics - which can be "modeled on a computer" \cite{uploading:1} (that computer can be Whole Brain Emulator, for example). This allowed him to create a simple "recipe" for transfering the mind from one brain to another:\\1. Scan every atom in the original brain (its location in 3D space, expressed by set of 3 coordinates
	\\2. Save the data about every single atom\\3. Move the data onto the new brain\\4. Launch a physics model simulation on the new brain (based on the position data of the atoms in the original brain)\cite{uploading:1}.\\While this doesn't sound complicated, there is one important question: how much storage would the it take to store the data about the entire human brain? 
	\\According to some early calculations presented by Merkle, to describe 1 atom we need on average around 100 bits. Human brain is estimated to be built from around 10\textsuperscript{26} atoms. This gives us a total of 10\textsuperscript{28} bits required to describe the brain \cite{uploading:1}. We are talking about around 10\textsuperscript{14} \emph{terabytes} of data. This sounds impossible as of right now, but it might not be in the future. According to statistics, data center storage capacity in 2021 came to the level of around 2300 \emph{exabytes} \cite{statista:capacity}. This is roughly 2,3*10\textsuperscript{9} \emph{terabytes}. It is sufficient to say that storage technology is slowly reaching the capacity required for storage of a human brain in a form of data about atoms.\\


\section{AI and the human evolution}
\subsection{What is the AI anyway?}
	\emph{"Can machines think?"}. This is the beginning of Alan Turing's paper dating back to 1950, in which he speculated that creation of thinking machines might be possible in the future \cite{turing}. Now, in year 2021, it is becoming a reality - todo: google itp

\subsection{The AGI}
	There are many ways to categorize AI, but one of the most general distinctions is \emph{weak AI} (\emph{narrow AI}) and \emph{strong AI} (often called \emph{Artifical General Intelligence}). The main difference is that the latter is supposed to be capable of performing one task, while the latter is supposed to be able to grasp anything that a human can learn or understand \cite{wikipediaAgi}. Obiously, the AGI is the thing that transhumanists focus on - a creation as smart as we are 
\subsection{The Singularity}
todo(finish)
\subsection{The Global Brain}
	todo(finish)
	
\section{Summary and discussion} 
	In this paper I have shown some of the ideas that have been advocated for by (modern) transhumanists. Those proposals sound as if they were taken from "sci-fi" novels, but in reality there are many scientists who are already researching them. For the purpose of this paper, I have decided to focus on possible enhancements of human mind through technology and AI. In my opinion, the first step of evolving into posthumans will be bypassing the limitations given to us by nature - especially things like brain's vulnerability to aging or permanent damage. This is why I have described Substrate-Independent Mind, Whole Brain Emulation, Artificial General Intelligence, The Singularity and the Global Brain ideas. All of them, combined, are possible alternatives and upgrades for a fragile powerhouse of our brain. There are still aspects that I haven't covered - such as moral implications or enhancements to the rest of human body. 
	\\In my opinion, intresting times are ahead of us. Quite literally, history is being made right in front of us - technological progress has been rapidly changing the world around us over the last 100 years and rate of progress is expected to accelerate even further. This is both exciting and dangerous - we won't know the outcome of those changes until they have actually happend (and by then it will be too late to reverse them). 
\section{todo}
todo: whole brain emulation
\raggedright
\bibliography{transhumanism-references}
\bibliographystyle{unsrt}
\end{document}